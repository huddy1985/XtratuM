\section{Inter-Domain Communication Tools}
\subsection{XM/FIFO}
\subsubsection{About XM/FIFO}
Fifo stands for "first in first out". It is a tool for communication. The fifo have name and a fixed storage. We can think of it as such: there is pipe to transmission of data stream. We write data to one port, and read data from the other port. The data which have been read can not been read again. If the write port reached to the read port, no data can write to the fifo again. At this time, we call the fifo is full. If the read port reached to the write port, no data can read from the fifo. At this time, we call the fifo is empty. If you want to know more about fifo, you can refer to the wiki:http://en.wikipedia.org/wiki/FIFO\_(computing). 
\\
We employ fifo in XtratuM, then we get the XtratuM/FIFO. Why the fifo is used in XtratuM or what is used for? If you have read the introduction about XtratuM, you should know the design of XtratuM. In order to raise the safety of the domains, XtratuM make the domains independent. Each domain has memory pages of itself. One domain can not access the the data of another domain directly, and these operations are forbid. However, some domains need to cooperate to finish some tasks. In order to improve the ability of coopertation, some comminucation tools should be used. The fifo can achieve these requirement and its realization is simple. 
\\
Take a simple example: domain d1 generates some data, and domain d2 deals with these data. The operations of d2 ought to be after the data generated in d1. How can d2 get the data from d1. It can realize simply with the help of fifo. When the data was generated, d1 write it to fifo; while d2 read these data from the fifo.
\subsubsection{Characteristics}
There are 16 fifos in XtratuM, named rtf0, rtf1, ... , rtf15. Fifo can be accessed by its name. These 16 fifos locale at /dev/, so they can be used by accessing /dev/rtf0, /dev/rtf1 and so on. In XtratuM, every domain can access these fifos: open, write, read, close. And these operations are similar to the operations of ordinary file in Linux.
\\
The storage size of each fifo is PAGE\_SIZE. The storages are assigned at the time of modules loaded, and they are in the memory all the time, unless some operations to unload these fifo modules. In order to manage or control these fifos, a struct fifo\_struct was used. A struct occupies 20-bits, including the fifo data memory address, the fifo can be used or not, the read point and write point. XtratuM use an array to organize the 16 structs. The opertations of array are simple. And it is not need to assign the memory to the struct when the fifo or domain is running.
\\
There are three parts in XtratuM/FIFO: FIFO core, Linux Fifo device drive, Partikle Fifo device drive. 
\\
The fifo core is belong to XtratuM, but separated from the XtratuM core. This part supports the fifo managment module. The fifo\_struct is used here. At the time of initialization, it registers sixteen fifo devices whose sizes are all PAGE\_SIZE. It also set the fifo\_struct of each fifo. At the time of end, the core of XM is in charge of exit of fifos. It will free the pages applied by fifos, and unregister these fifo devices. 
\\
When the domain wants to use these fifos, it should access the page of the fifo. the fifo core supports a function to do the memory map. The liner address of the domain can map to the page of the fifo, so it access the fifo by the linear address to read and write. In XtratuM, the pages mapped by the domain will free when the domain unload. This is not suit to the fifo pages. So a SHARE flag is used to mark the fifo page. When the domain is unloaded, the page can free except the fifo page. Just make the index in pte is 0.
\\
The core of XM support some function interfaces, the domain can use these fifos by these functions. While we konw that the core of XM support the fifo management, it registers some fifo devices. So the domain can use these devices directly by function of itself. Ok, now we can find that the different domains can use different methods to fifo. The Linux and Partikl are the examples.
\\
Operatation of access to the fifo is not blocked. When try to read from the fifo while the fifo is null, return directly. In the same way, when try to write the fifo while it is full, return directly. Each domain can read and write the fifo. The mechine of Lock-Free is used by the fifo to realize this goal. It use CAS(Compare-and-Swap) operation which make the read and write as atomic operation. You can read or write data successfully, otherwise, the data will not be changed. In other words, the operation can stop at any time. So it can avoid the operation block the task with high priority and it can enure the consistency of th e data.
\\
XM use two variant of fifo\_struct to realize read and write: ff\_top and ff\_buttom. The variant ff\_top points to the head of the fifo, used for write; and ff\_buttom point to the tail of the fifo, used for read. The CAS operation checks the values of ff\_top and ff\_buttom to check whether the read and write are interrupted or not.
\\
The CAS operation can avoid the competition of RR, RW,Wr. But unfortunately, the CAS operation can not avoid the competition of WW. So it is suggested that to reduce the number of task write to a fifo. 
\\
Linux fifo divece drive supports a way for the Linux domain to access the fifo. Linux is the root domain, and just use the function interfaces of the XM. The function of read is excutes in this flow:
read               Linux
lfifo\_read          Linux
rt\_fifo\_user\_read    XM
and the write is same.
\\
However, the Partikle domain can be loaded later, and can use the function of Partikle. The flow of function read executes like this:
\\
Although the implements are different, Linux and Partikle support some similar functions we can use, such as read, write, open, ioctl and so on.

\subsubsection{Using XM/FIFO}
In Partikle, when we create a domain using fifo, we should know the interrelated functions. As XM/FIFOs are devices too, so normal file operation routines can be used for FIFO devices, such open, close, read, write, etc..
\\
There are three examples about fifo in the directory: partikle/user/example/c\_example. Let's see their codes and results to know how the fifo works. \\

The first one is fifo.c . It uses varient fifo\_dev to point out that its will the fifo device- /dev/rtf0. In the content of the function, the fifo is open by the open() function:
\begin{verbatim}
if((fd = open(fifo_dev, O_RDWR, O_EXCL)) < 0)
\end{verbatim}
Now, the fd is the fifo descriptor of the fifo. We can write data to the fifo or read date from the fifo by it. In the example, it write a message to the fifo. The message is defined by varient msg.
\begin{verbatim}
char msg[] = "I am in the partikle\n";
... ...
for(i = 0; i < 10; i ++) {
    ret = write(fd, msg, sizeof(msg));
    printf("write fd %d ret = %d\n", fd, ret);
... ...
\end{verbatim}
Compile this domain, load it and display its message using that XtratuM tools.
\begin{verbatim}
$loader.xm fifo.prtk
$dmesg
write fd 3 ret = 22
... ...
write fd 3 ret = 22
\end{verbatim}

This is a simple example. The fifo is used for communicate inter-domain. Such example about communication between domain is supported, fwrite.c and fread.c. In this example, the fifo device /dev/rtf1 is used. So the two domain both open this fifo using function open. Fwrite domain write 50*2 digit string "1234567890" to rtf1, in the other port, fread domain read these 49 digits everytime. The digits written are uninterrupted, then the data read is interesting. The first time is 49 digits, just read to 9. The second time to read from 0, and end with 8. The code is this:
\begin{lstlisting}
char message[] = {"123456789012345678901234567890"};
... ...
for(i = 0; i < 50; i ++) {
ret = write(fd, message, 20);

char message[50];
... ...
message[49] = '\0';
for(i = 0; i < 5; i ++) {
ret = read(fd, message, 49);
... ...
printf("%d %d %s\n", ret, i, message);
\end{lstlisting}

Compile the two domain, and load them. Load the fwrite domain first to write data to fifo in order to ensure there is data in the fifo. Then display the message, you can get this:
\begin{verbatim}
$dmesg:
... ...
49 0 1234567890123456789012345678901234567890123456789
49 1 0123456789012345678901234567890123456789012345678
49 2 9012345678901234567890123456789012345678901234567
49 3 8901234567890123456789012345678901234567890123456
49 4 7890123456789012345678901234567890123456789012345
\end{verbatim}

After the introduce the functions and some examples, you can try to write a domain yourself. The domain can put into the directory of user/example/c\_example/, and execute "make". Then you can get the domain end with .prtk. Of cource, you can use the new directory to make the test. For example, I mkdir a new name test\_fifo.
\begin{lstlisting}

$mkdir -p /path/to/test_fifo
$cd /path/to/test_fifo
$cp /path/to/rtlinux-4.0-rc1/partikle/config.mk .
$cp /path/to/rtlinux-4.0-rc1/partikle/mkfiles/mkfile-c .
$vim Makefile

$cat Makefile
SOURCES = $(wildcard *.c)
NAMES = $(basename $(SOURCES))
TARGETS = $(addsuffix .prtk,$(NAMES))
all: $(TARGETS)
include config.mk
include mkfiles/mkfile-c
clean:
        $(Q3)rm -f *.{prtk,dump,sym,size,hex,bin} *.ktr
\end{lstlisting}
Write your program, compile and load it:

\begin{verbatim}
$vim fifo_test.c        //write the program of your self
$make 
$loader.xm fifo_test.prtk
\end{verbatim}
Note: You should remember that we use "open()" function to open the fifo decives. We should make the name is correct: 
/dev/rtf0 , /dev/rtf2 , ..., /dev/rtf15. If the name is just rtf0, rtf1, or rtf15, it will be wrong. As we said above, 
there are just SIXTEEN fifo device supported by XtratuM. So the /dev/rtf16, /dev/rtf17 and so on do not exist. You don't need try to use them. In patikle of XtratuM, the "open" function just is useful for fifo devices. If you tried to use it open a ordinary, it is not success.
The fifo is used for communication, so when you write a program, you had better plan which domain to write and which domain to read. The max data can write once is a PAGE\_SIZE, which is setted in the Linux Kernel.

\subsection{XM/SHM}
Shared memory is a method of interprocess communication (IPC) where two or more processes share a single chunk of memory to communicate. The shared memory system can also be used to set permissions on memory, allowing for things like malloc debuggers to be written. It is one of efficeint ways of IPC. This partion include XM/SHM, TLSF, APIs and Using examples.
\subsubsection{About XM/SHM}
Shared Memory in Xtratum is like a segment of memory, which can be mapped into address space of different domains of XtratuM including Linux and Partikle. So users can access any part of the segment in both real-time domain and non-real-time domain.
\\
Shared memory management module is the important part in the share memory system, as all the high level virtual memory must depend on it. It manages the physical memory, allocate or release the physical memory for operating systems run above XtratuM and mapped it into their address space. In the XtratuM system, the physical memory is managed by the Linux kernel, so we can use the Linux kernel APIs or some other functions excluding XtratuM APIs to allocate physical memory. 
\\
TLSF stands for Two Levels Segregate Fit memory allocator. It is designed for real-time system as a general purpose dynamic memory allocator. As the requirements of real-time system, the longest time to allocate memory and the conditions which will cause deadlock should be known and unrelated with the applications. This is the most important requirement for dynamic memory allocate in real-time system. Bounded response time is only a basic requirement, but not enough. According the application area of real-time, the response of time is required as short as possible. Besides the time requirement, real-time systems also need a guarantee of execution resource including memory. So TLSF should minimize the chances of exhausting the memory pool by minimizing fragmentation and wasted memory. \\
XM/SHM use two function malloc\_ex() and free\_ex() which are defined in xtratum/kernel/tlsf.c to alloc and free physical memory. 
\\
There are some features of SHM follow:
\begin{itemize}
\item Shared memory is typically used to hold a data structure whose members are read and written by two different processes.
\item Unlike FIFOs, shared memory is not queued. Any data written to shared memory overwrites the previous data.
\item Shared memory can also be accessed piecewise. Individual members of a large shared memory structure can be read and written independently.
\item Because it is a shared resource, care must be taken to ensure data consistency. In RT Linux, a real-time task can interrupt a Linux processes, but not vice-versa. There are two cases for a reader-writer pair:
\begin{itemize}
\item the RT task is the reader, and it interrupts the Linux process during its write. Here, the RT task will see an inconsistent data structure, with fresh data at the beginning and stale data at the end.
\item The RT task is the writer, and it interrupts the Linux process during its read. Here, the Linux process will see an inconsistent data structure, with stale data at the beginning and fresh data at the end.
\end{itemize}
\item A typical usage is a single RT Linux task writes a data array to shared memory periodically, and a single Linux process reads this data periodically and prints the results. Various types of data consistency techniques are demonstrated, so usually SHM accompany with other IPC tools like semaphore for inter-processes commumication. 
\end{itemize}
\subsubsection{XtratuM/SHM APIs}
You can access device of /dev/shm[0-15] after load XM/SHM modules. Every domain of XtratuM or Partikle can write and read the shm memory if it have access permission. So processes of Linux can communicate with processes of Partikle or other domain's.
\\
The most efficient way to implement shared memory applications is to rely on the mmap() function and on the system's native virtual memory facility. The process of Partikle domain use mmap() to alloc memory from device /dev/shm. Normally, the processes of XtratuM domain use functions mbuff\_alloc() and mbuff\_free() which are defined in ashm.h. mbuff\_alloc() call mmap() to allocate memeory and mbuff\_free() call munmap() to free memory.
\begin{itemize}
\item char *mbuff\_alloc(const char *name, ssize\_t size); 
\\
DESCRIPTION: alloc physical memory in shm device, name is a const string means the shm device name; size means the size of memory wait to alloc.
\\
RETURN VALUE: On success, mbuff\_alloc() return the pointer of alloced memory address. On error the NULL is returned.
\\
\item int mbuff\_free(char *start, int size\\
DESCRIPTION: free physical memory, start is pointer of string. size means the size of memory wait to free.
\\
RETURN VALUE: On success, mbuff\_free() returns 0. On error -1 (and sets errno) is returned.
\end{itemize}

\subsubsection{Usage Example}
In XMeRTL system, XM/SHM can be used to transfer data between normal Linux tasks, between PaRTiKle tasks, or between PaRTiKle task and Linux task. In the subclause, an example which use XM/SHM to transfer data between Linux and PaRTiKle will be showed. 
\\
The example comprised of two programs, prtk\_writer and linux\_reader. prtk\_writer is a PaRTiKle task which will write a string to a XM/SHM deivce. And linux\_reader will read the string from Linux space. Let's check the prtk\_writer.c file first.
\begin{lstlisting}
# cat prtk_writer.c



\end{lstlisting}
In the program, the prtk\_writer will use /dev/shm/0 to transfer data to linux\_reader. The device /dev/shm/0 will be mapped to /dev/shm0 in Linux system. So you will find that linux\_reader will open /dev/shm0 device. 
\\
explain the source code ......
\\
In order to compile the prtk\_writer successful, you should edit the Makefile like this.
\begin{lstlisting}
# cat Makefile

\end{lstlisting}

The prtk\_writer is ok now. Before compile the prtk\_writer.c file, linux\_reader.c should be done. The program can read the string from the shm device and output it to console.  

\begin{lstlisting}
# cat linux_reader.c


\end{lstlisting}
The linux\_reader.c is very simple. From the source code, it is same as normal linux programm. And it's clear that the shm device name is different. When the shm device is used, the device suffix number should be cared. In the exmple, /dev/shm0 and dev/shm/0 will present same device as their suffixes are 0. The Makefile for linux\_reader is simple.
\begin{lstlisting}
# cat Makefile

\end{lstlisting}
By far, all the source code for the example are finished. Let's compile, execute the example and see the result.
\begin{verbatim}

# cd prtk_writer_path
# make
# make -C linux_reader_path
# loader.xm ...
# linux_reader_path/linux_reader
....

\end{verbatim}

If you will see "Message from PaRTiKle!", it meanings your example is successful or failure. 
\\

Notes: before executing the example you should insure all XMeRTL has been installed successfully and shm devices have been created.
